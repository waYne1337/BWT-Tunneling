\documentclass{scrartcl}
\usepackage[utf8]{inputenc}
\usepackage[T1]{fontenc}
\usepackage{amsmath}
\usepackage{amssymb}
\usepackage{booktabs}
\usepackage{mathtools}
\usepackage{multirow}
\usepackage[dvipsnames]{xcolor}
\usepackage{pgfplots}
\usepackage{pgfplotstable}
\usepackage{longtable}
\usepackage{threeparttable}
\usepackage{ifthen}
\usepackage{url}
\usepackage{hyperref}
\usepackage{expl3}
\ExplSyntaxOn
\cs_new_eq:NN \fpeval \fp_eval:n
\ExplSyntaxOff
\pgfplotsset{compat=1.13}
\usetikzlibrary{patterns}

\definecolor{webgreen}{rgb}{0,0.5,0}

\title{\vspace{-5ex}De Bruijn graph edge reduction benchmark results}
\author{}
\date{}
\begin{document}
\maketitle

\vspace{-10ex}
\centering

%%%% matrix plot %%%%
\begin{figure}[h!t]
\centering
\pgfplotstablegetrowsof{edgeinfo.dat}
\edef\filenummatrixplot{\pgfplotsretval}
\begin{tikzpicture}
\begin{axis}[y=2ex,width=.85\textwidth,
	xlabel={\tiny Order $k$},
	xticklabel style={font=\tiny},
	xticklabel pos=top, xlabel near ticks,
	xmin=1,xmax=100,
	yticklabel style={font=\tiny},
	yticklabel={\pgfmathparse{int(\tick-1)}\pgfplotstablegetelem{\pgfmathresult}{File}\of{edgeinfo.dat}\tiny\pgfplotsretval},
	ytick={1,...,\filenummatrixplot},
	ymin=.5,ymax=\filenummatrixplot.5,
	y dir=reverse,
	colormap={redyellow}{rgb255(0cm)=(255,255,255); cmyk(2cm)=(0, 0.87, 0.68, 0.32)},
	colorbar,
	colorbar style={
		ytick={0,1},
		yticklabels={{\tiny \(\substack{\text{least}\\\text{edges}}\)},{\tiny \(\substack{\text{most}\\\text{edges}}\)}},
		height=20ex
	},
	enlargelimits=false,
	axis on top]
	\addplot [matrix plot*,point meta=explicit] file [meta=index 2] {edgematrix.dat};
\end{axis}
\end{tikzpicture}
\caption{Dependence between de Bruijn graph order and number of edges in an edge-reduced
	de Bruijn graph.}
\label{fig:dbgorderedgedependence}
\end{figure}

%%%% amount of reduced edges %%%%

\begin{figure}[ht]
\centering
\pgfplotstablegetrowsof{benchmark.dat}
\edef\filenumedgereduceplot{\pgfplotsretval}
\pgfplotstableset{
	create on use/min-edge-percentage/.style={
		create col/expr={\thisrow{min_dbg_edges}*100/\thisrow{input_length}} %later is added for labels
	},
	create on use/min-fm-percentage/.style={
		create col/expr={\thisrow{tfm_index_size}*100/\thisrow{fm_index_size}}
	},
	create on use/min-fm-overhead/.style={
		create col/expr={(\thisrow{tfm_index_size}*100/\thisrow{fm_index_size})-(\thisrow{min_dbg_edges}*100/\thisrow{input_length})}
	}
}
\begin{tikzpicture}
\begin{axis}[y=2ex,width=.85\textwidth,
	ylabel={\tiny best order $k^*$},
	yticklabel style={font=\tiny},
	yticklabel pos=right, ylabel near ticks,
	yticklabels from table={benchmark.dat}{min_dbg_k},
	ytick=data,
	xmin=1,	xmax=170,
	xtick=\empty,
	bar width=2ex,
	ymin=-.5,ymax=\filenumedgereduceplot-.5,
	y dir=reverse]
\addplot+[xbar,draw=none,fill=none] table [x expr=-5, y expr=\coordindex] {benchmark.dat};
\end{axis}
\begin{axis}[y=2ex,width=.85\textwidth,
	xlabel={\tiny Percentage of original DBG edges respectively original FM-index size},
	xticklabel style={font=\tiny},
	xticklabel={\pgfmathprintnumber\tick\%},
	xticklabel pos=top, xlabel near ticks,
	xmin=1,	xmax=170,
	xbar stacked,
	xmajorgrids=true,	
	extra x ticks=100,
	extra x tick labels={},
	extra x tick style={grid style={black,dashed,very thick}},
	bar width=2ex,
	yticklabel style={font=\tiny},
	yticklabels from table={benchmark.dat}{file},
	ytick=data,
	ymin=-.5,ymax=\filenumedgereduceplot-.5,
	y dir=reverse,
	ymajorgrids=true,
	legend style={legend columns=2,draw=none,at={(0.5,-1ex)},anchor=north},
	area legend,
	grid style=dashed]
\addplot+[xbar,draw=black,fill=Maroon!30] table [x=min-edge-percentage, y expr=\coordindex] {benchmark.dat};
\addplot+[xbar,draw=black,fill=RoyalBlue!30] table [x=min-fm-overhead, y expr=\coordindex] {benchmark.dat};
\legend{\parbox{12em}{\tiny edge count in minimal DBG},\parbox{12em}{\tiny tunneled FM-index size}}
\end{axis}
\end{tikzpicture}
\caption{Amount of reduced edges for minimal edge-reduced de Bruijn graphs as well as
	size of corresponding tunneled FM index compared to a normal FM-index.
	The overhead between the amount of reduced edges and the tunneled FM-index size comes
	from the two additional bitvectors required in the tunneled FM index.}
\label{fig:dbgreducededgesexperiments}
\end{figure}

%%%% required construction runtime %%%%

\begin{figure}[ht]
\centering
\pgfplotstablegetrowsof{benchmark.dat}
\edef\filedbgtimingsplot{\pgfplotsretval}
\pgfplotstableset{
	create on use/fm-time-percentage/.style={
		create col/expr={\thisrow{fm_time}*100/(\thisrow{fm_time} + \thisrow{min_dbg_time} + \thisrow{tfm_time})}
	},
	create on use/min-dbg-time-percentage/.style={
		create col/expr={\thisrow{min_dbg_time}*100/(\thisrow{fm_time} + \thisrow{min_dbg_time} + \thisrow{tfm_time})}
	},
	create on use/tfm-time-percentage/.style={
		create col/expr={(\thisrow{tfm_time}*100/(\thisrow{fm_time} + \thisrow{min_dbg_time} + \thisrow{tfm_time})}
	},
	create on use/tfm-total-time-seconds/.style={
		create col/expr={\fpeval{round(((\thisrow{fm_time} + \thisrow{min_dbg_time} + \thisrow{tfm_time}) / 1000)*1000)/1000}}
	}
}
\begin{tikzpicture}
\begin{axis}[y=2ex,width=.85\textwidth,
	ylabel={\tiny total time in seconds},
	yticklabel style={font=\tiny},
	yticklabel pos=right, ylabel near ticks,
	yticklabels from table={benchmark.dat}{tfm-total-time-seconds},
	ytick=data,
	xmin=1,	xmax=100,
	xtick=\empty,
	bar width=2ex,
	ymin=-.5,ymax=\filedbgtimingsplot-.5,
	y dir=reverse]
\addplot+[xbar,draw=none,fill=none] table [x expr=-5, y expr=\coordindex] {benchmark.dat};
\end{axis}
\begin{axis}[y=2ex,width=.85\textwidth,
	xlabel={\tiny Percentage of construction time},
	xticklabel style={font=\tiny},
	xticklabel={\pgfmathprintnumber\tick\%},
	xticklabel pos=top, xlabel near ticks,
	xmin=1,	xmax=100,
	xbar stacked,
	bar width=2ex,
	yticklabel style={font=\tiny},
	yticklabels from table={benchmark.dat}{file},
	ytick=data,
	ymin=-.5,ymax=\filedbgtimingsplot-.5,
	y dir=reverse,
	legend style={legend columns=3,draw=none,at={(0.5,-1ex)},anchor=north},
	area legend,
	axis on top]   
\addplot+[xbar,draw=black,fill=Maroon!30] table [x=fm-time-percentage, y expr=\coordindex] {benchmark.dat};
\addplot+[xbar,draw=black,fill=RoyalBlue!30] table [x=min-dbg-time-percentage, y expr=\coordindex] {benchmark.dat};
\addplot+[xbar,draw=black,fill=webgreen] table [x=tfm-time-percentage, y expr=\coordindex] {benchmark.dat};
\legend{\parbox{7.5em}{\tiny FM index},\parbox{7.5em}{\tiny DBG edge minimization},\parbox{7.5em}{\tiny tunneled FM-index}}
\end{axis}
\end{tikzpicture}
\caption{Construction timings of tunneled FM-index construction broken down into
	FM-index construction, de Bruijn graph edge minimization and tunneled FM-index construction.}
\label{fig:tfmconstructtimingsexperiments}
\end{figure}

%%%% required memory peak %%%%

\begin{figure}[ht]
\centering
\pgfplotstablegetrowsof{benchmark.dat}
\edef\filedbgmempeakplot{\pgfplotsretval}
\pgfplotstableset{
	create on use/tfm-fmc-mempeak-bytepersymbol/.style={
		create col/expr={\thisrow{fm_mem_peak}/\thisrow{input_length}}
	},
	create on use/tfm-mindbg-mempeak-bytepersymbol/.style={
		create col/expr={\thisrow{min_dbg_mem_peak}/\thisrow{input_length}}
	},
	create on use/tfm-tfmc-mempeak-bytepersymbol/.style={
		create col/expr={\thisrow{tfm_mem_peak}/\thisrow{input_length}}
	},
	create on use/tfm-mempeak-mbyte/.style={
		create col/expr={\fpeval{round((max(\thisrow{fm_mem_peak},max(\thisrow{min_dbg_mem_peak},\thisrow{tfm_mem_peak}))/(1024*1024))*1000)/1000}}
	}
}
\begin{tikzpicture}
\begin{axis}[y=2ex,width=.85\textwidth,
	ylabel={\tiny memory peak in MiB},
	yticklabel style={font=\tiny},
	yticklabel pos=right, ylabel near ticks,
	yticklabels from table={benchmark.dat}{tfm-mempeak-mbyte},
	ytick=data,
	xmin=1,	xmax=10,
	xtick=\empty,
	bar width=2ex,
	ymin=-.5,ymax=\filedbgmempeakplot-.5,
	y dir=reverse]
\addplot+[xbar,draw=none,fill=none] table [x expr=-5, y expr=\coordindex] {benchmark.dat};
\end{axis}
\begin{axis}[y=2ex,width=.85\textwidth,
	xlabel={\tiny Memory peak in bytes per input symbol},
	xticklabel style={font=\tiny},
	xticklabel pos=top, xlabel near ticks,
	xmin=1,	xmax=10,
	no markers,
	xmajorgrids=true,
	bar width=.4ex,
	yticklabel style={font=\tiny},
	yticklabels from table={benchmark.dat}{file},
	ytick=data,
	ymin=-.5,ymax=\filedbgmempeakplot-.5,
	y dir=reverse,
	ymajorgrids=true,
	grid style=dashed,
	legend style={legend columns=3,no markers,draw=none,at={(0.5,-1ex)},anchor=north},
	area legend]   
\addplot+[bar shift=.4ex, xbar,draw=black,ultra thin,fill=Maroon!30] table [x=tfm-fmc-mempeak-bytepersymbol, y expr=\coordindex] {benchmark.dat};
\addplot+[bar shift=0ex,  xbar,draw=black,ultra thin,fill=RoyalBlue!30] table [x=tfm-mindbg-mempeak-bytepersymbol, y expr=\coordindex] {benchmark.dat};
\addplot+[bar shift=-.4ex,xbar,draw=black,ultra thin,fill=webgreen] table [x=tfm-tfmc-mempeak-bytepersymbol, y expr=\coordindex] {benchmark.dat};
\legend{\parbox{7.5em}{\tiny FM index},\parbox{7.5em}{\tiny DBG edge minimization},\parbox{7.5em}{\tiny tunneled FM-index}}
\end{axis}
\end{tikzpicture}
\caption{Memory peak during tunneled FM index construction, measured in bytes per symbol of the input data.}
\label{fig:tfmconstructmempeaksexperiments}
\end{figure}

\end{document}
